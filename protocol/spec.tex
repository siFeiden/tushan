\documentclass{scrartcl}

\usepackage[utf8]{inputenc}
\usepackage{amsmath}
\usepackage{csquotes}
\usepackage{tikz}
\usepackage{url}
\usepackage{mdframed}
\usepackage{multicol}
\usepackage[outputdir = build]{minted}
\setminted{fontsize = \footnotesize}

\usepackage{hyperref}

\usetikzlibrary{decorations.pathreplacing, angles, quotes, matrix, positioning,
  patterns}

\newcommand{\tushan}{\textsc{Tushan}}

\title{\tushan{}}
\author{Voltidioten}

\begin{document}

\maketitle

This document is the specification of \tushan{}. It contains the 
definitions of game rules, logical entities and their representation and 
protocol specification. The protocol relies on exchanging JSON-messages, hence
every entity of the game is encoded as such.

\section{Game Description}
\tushan{} is based on the game 
\href{https://en.wikipedia.org/wiki/Ta\_Y\%C3\%BC}{Ta-Yü}. Two players
oppose each other on a quadratic field which is separated in cells. Each player 
is assigned two opposite sites of the field. In turns the player place stones
which cover cells. The stones are equipped with connectors on its sites. These
connectors form a flow through the stones. The game ends if there are no valid
placements of the current stone and every player is awarded the product of two
factors, where each factor corresponds to the number of connectors immediately
touching the sites of the field (of said player). These notions are specified
in the following.

\section{Entities}

\subsection{Stone}
\hypertarget{stone}{}
The stones are placed by the players onto the field to form the flow and 
eventually determine the result of the game. A stone is a rectangular shape 
with a width $n$ and a height $m$. The notions of width and height are rastered 
by the cells, hence any stone covers an \hyperlink{area}{\emph{area}} of 
$n\times m$ cells. The number of connectors is variable although throughout one 
game it usually is consistent. There is at least one connector (although for a 
game with consistent connector numbers stones with one connector enforce 
trivial results). Stones are described in terms of $n$ and $m$ and their 
connectors. Additionally, to incorporate their placements in the game we opt to 
use a representation which contains one specific reference cell (the top left 
corner) and their orientation in terms of the direction the top edge is facing 
(given by one of the four values \mintinline{json}{"north"}, 
\mintinline{json}{"east"}, \mintinline{json}{"south"} or 
\mintinline{json}{"west"}). The connectors are identified by indices which move 
around clockwise (starting by $0$) from the top left covered cell.

\begin{mdframed}[frametitle = {Representation and Illustration}]
  The encoding of one stone might depend on the fact if it is already placed 
  or not. Regardless, we define for any $n,m > 0$ and $c_{0}, \dots, c_{n}$ 
  pairwise distinct values with $0 \leq c_{i} < 2\cdot n + 2\cdot m$ for 
  $0\leq i\leq n$.
  \begin{minted}[escapeinside=||]{text}
    {
      "width": |$n$|,
      "height": |$m$|,
      "connectors": [|$c_0, \dots, c_n$|]
    }
  \end{minted}
  If the stone is placed in a \hyperlink{field}{\emph{field}} this is 
  indicated by the inclusion of an additional object with key 
  \mintinline{json}{"position"}. Here valid ranges of coordinates 
  $x, y \geq 0$ depend on the \hyperlink{field}{\emph{field}} the stone is 
  placed in while $o$ is always one of the values \mintinline{json}{"north"}, 
  \mintinline{json}{"east"}, \mintinline{json}{"south"} or 
  \mintinline{json}{"west"}.
  \begin{minted}[escapeinside=||]{text}
    {
      "width": |$n$|,
      "height": |$m$|,
      "connectors": [|$c_0, \dots, c_n$|],
      "position": {
        "x": |$x$|,
        "y": |$y$|,
        "orientation": |$o$|
      }
    }
  \end{minted}
  \begin{center}
    \begin{tikzpicture}
      \matrix (stone) [
        matrix of nodes,
        nodes in empty cells,
        nodes = {
          draw, 
          rectangle,
          anchor = center,
          minimum height = 1.2cm,
          minimum width = 1.2cm,
        }
        ] {
           &  & \\
           &  & \\
           &  & \\
      };
      \draw [ decoration = { brace, amplitude = 5pt, raise = 20pt }, 
        decorate, line width = 1pt ]
          (stone-1-1.north west) -- node [above, yshift = 25pt] {$n$} 
          (stone-1-3.north east);
      \draw [ decoration = { brace, amplitude = 5pt, raise = 80pt }, 
        decorate, line width = 1pt ]
          (stone-3-1.south west) -- node [left, xshift = -85pt] {$m$} 
          (stone-1-1.north west);
      \draw [ decoration = { brace, amplitude = 5pt, raise = 50pt }, 
        decorate, line width = 1pt ]
          (stone-1-3.north east) -- node [right, xshift = 55pt] {$m$} 
          (stone-3-3.south east);
      \draw [ decoration = { brace, amplitude = 5pt, raise = 20pt }, 
        decorate, line width = 1pt ]
          (stone-3-3.south east) -- node [below, yshift = -25pt] {$n$} 
          (stone-3-1.south west);

      \node [yshift = 5pt]   at (stone-1-1.north){$0$};
      \node [yshift = 5pt]   at (stone-1-3.north){$n - 1$};
      \node [xshift = 5pt]   at (stone-1-3.east) {$n$};
      \node [xshift = 25pt]  at (stone-3-3.east) {$n + m - 1$};
      \node [yshift = -5pt]  at (stone-3-3.south){$n + m$};
      \node [yshift = -5pt]  at (stone-3-1.south){$2\cdot n + m - 1$};
      \node [xshift = -25pt] at (stone-3-1.west) {$2\cdot n + m$};
      \node [xshift = -40pt] at (stone-1-1.west) {$2\cdot n + 2\cdot m - 1$};

      \node [circle, draw, fill] (orientation start) at (stone-1-1.center){};
      \draw [->, line width = 1.5pt] (orientation start) to node [above, 
      xshift=-2pt] 
        {\tiny orientation} (stone-1-3.center);
    \end{tikzpicture}
  \end{center}
\end{mdframed}


\begin{mdframed}
  \hypertarget{area}{\emph{Area of placed stone}} - The area of a placed stone 
  is defined by all cells it covers, namely all cells with coordinates 
  \begin{equation*}
    (a, b) \text{ s. t. }
    \begin{cases}
      x\leq b\leq x + m - 1, y - n + 1\leq a\leq y &\text{if }o = \texttt{"north"}\\
      x\leq a\leq x + n - 1, y\leq b\leq y + m - 1 &\text{if }o = \texttt{"east"}\\
      x - m + 1\leq b\leq x, y\leq a\leq y + n - 1 &\text{if }o = \texttt{"south"}\\
      x - n + 1\leq a\leq x, y - m + 1\leq b\leq y &\text{if }o = \texttt{"west"}\\
    \end{cases}
  \end{equation*}
  where $o$ is the orientation of the stone and $x, y$ its respective $x$- and
  $y$-coordinate.
\end{mdframed}
\begin{mdframed}
  \hypertarget{initial}{\emph{Initial stone}} - A stone is called initial if it 
  is the only stone placed onto a \hyperlink{field}{\emph{field}}. The 
  \hyperlink{area}{\emph{area}} of any initial stone must contain one of the 
  following cells
  \begin{itemize}
    \item $(\frac{n}{2}, \frac{n}{2})$,
    \item $(\frac{n}{2} + 1, \frac{n}{2})$,
    \item $(\frac{n}{2}, \frac{n}{2} + 1)$,
    \item $(\frac{n}{2} + 1, \frac{n}{2} + 1)$.
  \end{itemize}
\end{mdframed}
\begin{mdframed}
  \hypertarget{valid}{\emph{Validity of placed stones}} - A stone is considered 
  validly placed if its \hyperlink{area}{\emph{area}} is inside the field; that 
  is if for all $(a, b)$ in its area $1\leq a, b\leq d$ for a 
  \hyperlink{field}{\emph{field}} with dimension $d$. Furthermore, it holds 
  that \emph{every} connector is connected either to one of the delimiting 
  walls, to another connector or an empty cell. Additionally, the stone is 
  either \hyperlink{initial}{\emph{initial}} or there is at least \emph{one} 
  connector that is adjacent to another connector.
\end{mdframed}

\begin{mdframed}[frametitle = {Example}]
  Not yet placed:\\
  \begin{minipage}{0.5\textwidth}
    \inputminted{json}{examples/not-placed-stone.json}
  \end{minipage}%
  \begin{minipage}{0.5\textwidth}
    \begin{center}
      \begin{tikzpicture}
        \matrix (stone) [
          matrix of nodes,
          nodes in empty cells,
          nodes = {
            draw, 
            rectangle,
            anchor = center,
            minimum height = 1.2cm,
            minimum width = 1.2cm,
          }
        ] {
           &  & \\
        };
        \node [fill = green!50, circle, draw] (c0) at (stone-1-1.north) {
          \tiny{0}};
        \node [fill = red!50, circle, draw]   (c1) at (stone-1-2.north) {
          \tiny{1}};
        \node [fill = red!50, circle, draw]   (c2) at (stone-1-3.north) {
          \tiny{2}};
        \node [fill = green!50, circle, draw] (c3) at (stone-1-3.east)  {
          \tiny{3}};
        \node [fill = red!50, circle, draw]   (c4) at (stone-1-3.south) {
          \tiny{4}};
        \node [fill = red!50, circle, draw]   (c5) at (stone-1-2.south) {
          \tiny{5}};
        \node [fill = red!50, circle, draw]   (c6) at (stone-1-1.south) {
          \tiny{6}};
        \node [fill = green!50, circle, draw] (c7) at (stone-1-1.west)  {
          \tiny{7}};

        \node [circle, draw, fill] (orientation start) at (stone-1-1.center) {};
        \draw [->, line width = 1.5pt] (orientation start) to node [above, 
        xshift=-2pt] 
          {\tiny orientation} (stone-1-3.center);
      \end{tikzpicture}
    \end{center}
    \tiny{Green circles indicate a connector while red circles mark positions of
    possible but for this stone not available connectors.}
  \end{minipage}
  Already placed:\\
  \begin{minipage}{0.5\textwidth}
    \inputminted{json}{examples/placed-stone.json}
  \end{minipage}%
  \begin{minipage}{0.5\textwidth}
    \begin{center}
      \begin{tikzpicture}
        \matrix (stone) [
          matrix of nodes,
          nodes in empty cells,
          nodes = {
            draw, 
            rectangle,
            anchor = center,
            minimum height = 1.2cm,
            minimum width = 1.2cm,
          }
        ] {
           \\
           \\
           \\
        };
        \node [fill = green!50, circle, draw] (c0) at (stone-1-1.east)  {
          \tiny{0}};
        \node [fill = green!50, circle, draw] (c0) at (stone-3-1.south) {
          \tiny{3}};
        \node [fill = green!50, circle, draw] (c0) at (stone-1-1.north) {
          \tiny{7}};

        \node [circle, draw, fill] (orientation start) at (stone-1-1.center) {};
        \draw [->, line width = 1.5pt] (orientation start) to (stone-3-1.center);

        \node [left  = 2 of orientation start] (ystart) {$y = 3$};
        \node [right = 2 of orientation start] (yend)   {};

        \node [above = 2 of orientation start] (xstart) {$x = 1$};
        \node [below = 3.4 of orientation start] (xend)   {};

        \draw [dashed] (ystart) to (yend);
        \draw [dashed] (xstart) to (xend);
      \end{tikzpicture}
    \end{center}
  \end{minipage}%
\end{mdframed}


\subsection{Field}
\hypertarget{field}{}
The field of the game is a quadratic collection of cells. Each cell is 
identified by its coordinates starting with $(0,0)$ at the top left and 
proceeding to the left with increasing first element and to the bottom with 
increasing second element, this corresponds to a discrete cartesian coordinate 
system starting at the top left growing right- and downwards. The four values 
\enquote{north}, \enquote{east}, \enquote{south} and \enquote{west} refer to 
the top, left, bottom and right side respectively and are especially relevant 
for the orientation of placed \hyperlink{stone}{\emph{stones}}.

\begin{mdframed}[frametitle = {Representation and Illustration}]
  The parameter of a field are its dimension $d$ and placed 
  \hyperlink{stone}{\emph{stones}}. Therefore, we define for an even $d > 1$ 
  and \hyperlink{valid}{\emph{validly placed}} \hyperlink{stone}{\emph{stones}}
  $s_{1}, \dots, s_{k}$ the representation of a field as
  \begin{minted}[escapeinside=||]{text}
    {
      "dimension": |$d$|,
      "stones": [|$s_0, \dots, s_k$|]
    }
  \end{minted}
  \begin{center}
    \begin{tikzpicture}
      \matrix (field) [
        matrix of nodes,
        column sep = 0pt,
        row sep = 0pt,
        nodes in empty cells,
        inner xsep = 0pt,
        nodes = {
          rectangle, draw, minimum width = 1.2cm, minimum height = 1.2cm, 
          outer sep = 0pt, inner xsep = 0pt, inner ysep = 0pt, anchor = center
        }] {
          $(0,0) $ & $(1,0) $ & $\dots $ & $(d,0) $ \\
          $(0,1) $ & $(1,1) $ & $\dots $ & $(d,1) $ \\
          $\vdots$ & $\vdots$ & $\ddots$ & $\vdots$ \\
          $(0,d) $ & $\dots $ & $\dots $ & $(d,d) $ \\
      };
      \draw [ decoration = { brace, raise = 15pt, amplitude = 10pt }, decorate, 
        line width = 1pt ]
          (field-1-1.north west) -- node [above, yshift = 25pt] {Player $A$} 
          (field-1-4.north east);
      \draw [ decoration = { brace, raise = 5pt, amplitude = 10pt, mirror }, 
        decorate, line width = 1pt ]
          (field-4-1.south west) -- node [below, yshift = -15pt, label = {
            [rotate = 180] {Player $A$}}] {} (field-4-4.south east);
      \draw [ draw, -> , line width = 1pt] ([yshift = 5pt] field-1-1.north west) 
        to node [above] {$x$} ([yshift = 5pt] field-1-4.north east);
      \draw [ draw, -> , line width = 1pt] ([xshift = -5pt] field-1-1.north west) 
        to node [left] {$y$} ([xshift = -5pt] field-4-1.south west);
      \draw [ decoration = { brace, raise = 15pt, amplitude = 10pt, mirror }, 
        decorate, line width = 1pt ]
          (field-1-1.north west) -- node [left, xshift = -25pt, label = {
            [rotate = 90] {Player $B$}}] {} (field-4-1.south west);
      \draw [ decoration = { brace, raise = 5pt, amplitude = 10pt }, 
        decorate, line width = 1pt ]
          (field-1-4.north east) -- node [right, xshift = 15pt, label = {
            [rotate = -90] {Player $B$}}] {} 
          (field-4-4.south east);

      \node [above = 2 of field] (n) {north};
      \node [right = 1.6 of field, label = {[rotate = -90] {east}}] {};
      \node [below = 1.6 of field, label = {[rotate = 180] {south}}] {};
      \node [left  = 2 of field, label = {[rotate = 90] {west}}] {};
    \end{tikzpicture}
  \end{center}
\end{mdframed}

\begin{mdframed}[frametitle = {Example}]
  \begin{multicols}{2}
    \inputminted{json}{examples/populated-field.json}
  \end{multicols}
  \begin{center}
    \newcommand{\drawstone}[3][north west lines]{
      \draw [pattern = #1, pattern color = gray] 
        (#2.north west) rectangle (#3.south east)
    }
    \newcommand{\orientation}[2]{
      \node [draw, circle, fill = black] at (#1.center) {};
      \draw [->, line width = 1pt] (#1.center) to (#2.center)
    }
    \begin{tikzpicture}
      \matrix (field) [
        matrix of nodes,
        nodes in empty cells,
        nodes = {
          draw,
          dotted,
          rectangle,
          anchor = center,
          minimum height = 1.2cm,
          minimum width = 1.2cm,
        }
        ] {
           &  &  &  &  &  &  &  &  & \\
           &  &  &  &  &  &  &  &  & \\
           &  &  &  &  &  &  &  &  & \\
           &  &  &  &  &  &  &  &  & \\
           &  &  &  &  &  &  &  &  & \\
           &  &  &  &  &  &  &  &  & \\
           &  &  &  &  &  &  &  &  & \\
           &  &  &  &  &  &  &  &  & \\
           &  &  &  &  &  &  &  &  & \\
           &  &  &  &  &  &  &  &  & \\
      };

      \drawstone{field-1-1}{field-1-2};
      \orientation{field-1-2}{field-1-1};
      \drawstone{field-2-2}{field-3-4};
      \orientation{field-2-2}{field-2-4};
      \drawstone[north east lines]{field-2-5}{field-5-5};
      \orientation{field-5-5}{field-2-5};
      \drawstone{field-5-6}{field-6-7};
      \orientation{field-5-6}{field-5-7};
      \drawstone{field-3-6}{field-3-8};
      \orientation{field-3-6}{field-3-8};

      \node [circle, fill = green!50] at (field-1-1.north) {};
      \node [circle, fill = green!50] at (field-1-2.north) {};
      \node [circle, fill = green!50] at (field-2-2.north) {};
      \node [circle, fill = green!50] at (field-2-2.west)  {};
      \node [circle, fill = green!50] at (field-3-4.east)  {};
      \node [circle, fill = green!50] at (field-3-5.east)  {};
      \node [circle, fill = green!50] at (field-3-7.north) {};
      \node [circle, fill = green!50] at (field-3-8.north) {};
      \node [circle, fill = green!50] at (field-5-5.east)  {};
      \node [circle, fill = green!50] at (field-6-6.south) {};
      \node [circle, fill = green!50] at (field-6-7.south) {};
      \node [circle, fill = green!50] at (field-6-7.east)  {};

    \end{tikzpicture}
  \end{center}
\end{mdframed}

\end{document}
